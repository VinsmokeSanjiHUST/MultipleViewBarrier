\documentclass[3p]{elsarticle}

\usepackage{tgpagella}

\usepackage{hyperref}
\hypersetup{
	colorlinks=true,
	linkcolor=blue,
	filecolor=magenta,      
	urlcolor=black,
}
\usepackage[table, dvipsnames]{xcolor}
\usepackage{graphicx}
\usepackage[linesnumbered, ruled, longend]{algorithm2e}
\usepackage{amsmath, amssymb}

%%%%%%%%%%%%%%%%%%%%%%%
%% Elsevier bibliography styles
%%%%%%%%%%%%%%%%%%%%%%%
%% To change the style, put a % in front of the second line of the current style and
%% remove the % from the second line of the style you would like to use.
%%%%%%%%%%%%%%%%%%%%%%%

%% Numbered
%\bibliographystyle{model1-num-names}

%% Numbered without titles
%\bibliographystyle{model1a-num-names}

%% Harvard
%\bibliographystyle{model2-names.bst}\biboptions{authoryear}

%% Vancouver numbered
%\usepackage{numcompress}\bibliographystyle{model3-num-names}

%% Vancouver name/year
%\usepackage{numcompress}\bibliographystyle{model4-names}\biboptions{authoryear}

%% APA style
%\bibliographystyle{model5-names}\biboptions{authoryear}

%% AMA style
%\usepackage{numcompress}\bibliographystyle{model6-num-names}

%% `Elsevier LaTeX' style
\bibliographystyle{elsarticle-num}
%%%%%%%%%%%%%%%%%%%%%%%

%%%%%%%%%%%%%%%%%%%%%%%%
\setlength{\parindent}{10pt}
\setlength{\parskip}{6pt}
%%%%%%%%%%%%%%%%%%%%%%%%
\newtheorem{thr}{Theorem}
\newdefinition{df}{Definition}
\newproof{pf}{Proof}
%%%%%%%%%%%%%%%%%%%%%%%%
\usepackage{color}
\definecolor{light-gray}{gray}{0.95}
%%%%%%%%%%%%%%%%%%%%%%%%
\usepackage{scalerel}
\DeclareMathOperator*{\uint}{\scalerel*{\rotatebox{8}{$\!\scriptstyle\int\!$}}{\int}}

\begin{document}

\begin{frontmatter}

%\title{An efficient approximate algorithm for achieving $(k-\omega)$ angle barrier coverage in camera wireless sensor networks}
\title{An efficient method to verify and assess ($k-\omega$) coverage in wireless multimedia sensor networks}

%% Group authors per affiliation:
\author[1]{Nguyen Thi My Binh}
\ead{binhntm@haui.edu.vn}
\author[2]{Chu Minh Thang}
\ead{mthang129@gmail.com}
\author[3]{Huynh Thi Thanh Binh}
\ead{binhhtt@soict.hust.edu.vn}

\begin{abstract}
Barrier coverage problems in camera sensor networks have drawn the attention by academic community because of their huge potential applications. Various versions of barrier coverage under wireless camera sensor networks have been studied such as minimal exposure path, strong/weak barrier, 1/k barrier, full view barrier problems. In this paper, based on new $(k-\omega)$ angle coverage model, we study how to achieve $(k-\omega)$ angle barrier coverage problem under uniform random deployment scheme (hereinafter $A(k-\omega)ABC$ problem). This problem aims to juggle whether any given camera sensor networks is $(k-\omega)$ angle barrier coverage. A camera sensor network is called $(k-\omega)$ angle barrier coverage if any crossing path is $(k-\omega)$ angle coverage. The $A(k-\omega)ABC$ problem is useful because it can make balance of the number of camera sensors used and the information retrieved by the camera sensors. Furthermore, this problem is vital for design and applications for camera sensor networks when camera sensor nodes were deployed randomly. Thus, we formulate the $A(k-\omega)ABC$ problem and then proposed an efficient method for solving this problem. An extensive experiments were conducted on random instances, and the results indicated that the proposed algorithm can produce high quality and stable solutions, as well as achieve better results than full-view coverage model.
\end{abstract}

\begin{keyword}
minimal exposure path\sep wireless sensor networks\sep stationary sensor networks\sep mobile sensor networks
\end{keyword}

\end{frontmatter}

%\linenumbers

\section{Introduction}
\label{sec:intro}
Barrier coverage in WSNs is a critical issue for various sensor network security applications, e.g., boundary surveillance (national border, critical resource protection) and intrusion detection. Barrier coverage is formed by a sensing barrier for detecting intruders crossing a region of interest. Compared with full coverage, barrier coverage requires much less quantity of sensors and energy. Thus, barrier coverage is considered more scalable and attractive in large-scale deployment in reality.\par
Recently, camera sensor networks have drawn the attention of research community [1-6]. Compared with conventional scalar sensors, camera sensors can harvest much richer information of the environment in the forms of images or videos and thus promise an extremely potential in applications. However, cost of camera sensors is higher than scalar sensor. Camera sensors are randomly scattered for achieving barrier coverage in a large scale is a very challenged. Thus, security surveillance or intrusion detection application is expected to build up a cost efficient camera sensor barrier such that every intruder image can be detected effectively. However, the camera sensors barrier coverage problem is much different and more complicated than the conventional barrier coverage problem. When sensing range of a chain of camera sensors across the surveillance region is simply combining, that does not provide effective barrier coverage. Because an intruder may cross the barrier without being recognized, i.e., its face image could not be caught. The fundamental difference between camera and traditional scalar sensors in coverage is that camera sensors may capture very different scenes of the same object if they are from different viewpoints. For instance, when a camera is placed behind the intruder, no face image can be recognized. Researches in the computer vision field show that the object is more likely to be recognized by the recognition system if the image is captured at or near the frontal viewpoint \cite{blanz2005face}, i.e., if the object is facing straight to the camera. As the angle between the object facing direction and the camera viewing direction (defined by the vector from the object to the camera, Figure 1(a)) increases, the detection rate decreases dramatically. Therefore, to keep a high level monitor quality, a good camera sensor barrier should ensure that no matter where the traversing object faces, there is always some camera to effectively capture its face image.\par
%
To get more information of the intruders, especially their face identifications, full-view coverage model was first introduced in Wang and Cao [8, 9]. An object is full-view covered if there is always at least one camera covering it no matter which object facing direction and the cameras viewing direction is sufficiently close to the object facing direction. Obviously, we can obtain more information about the intruder in full-view model. In \cite{wu2012achieving}, the authors investigate the necessary and sufficient conditions to achieve full view coverage under uniform deployment and Poisson deployment. \par
%
Ma et al. \cite{ma2012minimum} studied the minimum camera barrier coverage problem in WCSNs when the camera sensors are deployed randomly in a monitor field. The authors have some extension, the algorithm reduces the number of camera sensors, but it still requires a lot of camera sensors to construct a full viewed barrier coverage. In \cite{xu2016minimum}, Xu et. al. also want to maintain the quality of aggregate information with the least camera sensors. Basing on $(k-\omega)$ angle coverage model \cite{tseng2012k}, they produce a new barrier coverage model named by $(k, \omega)$-angle barrier coverage. The object is said $(k-\omega)$ angle coverage when it is simultaneously monitored by at least k sensors, and to ensure that k different sensors monitor different parts of the object, the angle between two arbitrary sensor directions must be equal or greater. The monitored field exists a $(k-\omega)$ angle barrier coverage if it has a connected region across the monitor field such that every point within the region is $(k-\omega)$ angle coverage. The $(k-\omega)$ angle barrier coverage is useful because of a balance of the number of camera sensors used and the information retrieved by the camera sensors.\par
%
However, a fundamental problem of assessing the quality of a $(k-\omega)$ angle barrier has not been addressed efficiently. Currently, in the tradition model of attenuation, when evaluating the coverage quality of a sensor toward a certain point, the coverage value formula is defined to be affected only by the distance from the sensor to the considered point, which may lead to several exceptional inconsistency regarding evaluate the quality of coverage, especially in the problem of $(k-\omega)$ barrier in the sensor network. Firstly, the model can only illustrate the general coverage of a point to a certain sensor. As a result, it is impossible to achieve a metric measuring the amount of information on every direction of an intruder captured by the network of sensors, which is fundamental in several coverage problems such as the $k-\omega$ cover or the full view cover problem. Secondly, this model shows little consistency considering the coverage of a point in a sensor network. In case of evaluating the coverage of a point exposed to numerous sensors, there have been 2 common models, the All-Sensor Field Intensity and the Closest-Sensor Field Intensity model. While the latter failed to illustrate the cooperation of the sensors in sensing the same object, which greatly underrate the real coverage of the network on the intruder, the former model does not take into account the repetition of information captured by numerous sensors, thus exaggerate the value of coverage dramatically.\par
%
As a result, we found it necessary to devise a more preferable attenuated model assessing the coverage quality of the sensor network toward a point in the field of interest, the model can later be generalised to evaluate the coverage of the sensor network on a line or a closed region. To formulate the coverage value regarding information from every direction of the intruder, it is necessary to consider its shape. In this article, the problem is tackled generally, which results in we using the rule of symmetric to make an assumption that the intruder is a circle. This geometrical shape is sufficient since the coverage is considered in the 2-D plane only, and the circle itself has parts of the circumference followed every possible direction, which help illustrate the devised model more thoroughly and clearly.\par
%
The coverage model implements the idea of Divide-And-Conquer. The intruder is differentiated into several small parts. Each of them is then evaluated separately, then added together to obtain the total coverage of the sensor network on the intruder at the considered position. Each small part of the circle, in turn, is calculated with each sensor that cover it, then the largest coverage value is taken. This operation will prevent the coverage overrated from several sensors having similar position toward the considered point, which in real life will serve no purpose of obtaining more information of the specific part on the intruder.\par
%
In this paper, we focus on evaluating a sensing field based on two factors: the probability of successfully forming a $(k-\omega)$ angle barrier, and the average quality of the found barriers in the sensing fields. This problem is vital for optimizing the efficiency of sensor placement in sensor networks when camera sensor nodes were deployed randomly. For instance, camera sensors have to be scattered by plane or artillery if the region of interest is hostile or inaccessible or deficiency on time, manpower or funds prevents careful arrangement of every single sensor.\par
%
The main contributions of this article are as follows. 
\begin{itemize}
	\item Formulate achieving ($k-\omega$) angle barrier coverage problem.
	\item Propose an efficient method to deterministically verify if a monitored field can be achieved a ($k-\omega$) angle barrier coverage in by any given set of camera sensors. 
	\item Evaluate the found barriers with the devised metric.
	\item Conduce experiments in various scenarios to examine the result and the computation time of the proposed algorithm.
	\item Analyze and estimate the parameters effect on performance of proposed algorithm.
\end{itemize}
%
The rest of the paper is organized as follow. Related works are presented in section 2. Section 3 formulates A ($k-\omega$) angle barrier problem. Section 4 introduces proposed algorithm. Section 5 gives our experiments along with computational and comparative results as well as conclusion in section 6.
%

\section{Related works}
The concept of barrier coverage \cite{kumar2005barrier}, was first introduced specifically for intruder detection applications in WSNs where sensing regions of sensor nodes form one or multiple barriers so that any intruder penetrating the region of interest will be detected. Due to its superiorities for security applications, barrier coverage has received attentions in recent years. The barrier coverage problems in WSNs can be categorized into two sub-problems: one is finding penetration paths. A penetration path is continuous curve with arbitrary shape, go though one side to the other side of a sensor field; other is building intrusion barrier for detecting intrusion of a mobile object when it traverse from on side to the other side of the sensing field. The first problems have thoroughly been delved into many researches such as \cite{megerian2002exposure,binh2016heuristic,liu2013percolation,binh2017genetic,binh2019efficient}, the second ones mostly focused on critical condition analysis (e.g., sensor node density) and barrier construction for stationary sensors with omni-directional sensing coverage models. \cite{liu2008strong,saipulla2008barrier,he2010distributed,skraba2007energy,chen2013energy}. Directional sensing coverage models then were widely used such as camera, radar etc., and taken into consideration in coverage problems as well as in barrier coverage problems.
\cite{ai2006coverage,akyildiz2007survey,guvensan2011coverage,ma2005coverage,soro2005coverage}. Barrier coverage problems in WCSNs are much more complexed and challenging compared to those in traditional scalar WSNs \cite{chang2006collaborative,ma2012minimum,makhoul2009adaptive,wang2011barrier}, because of WCSNs having unique features.
 
The authors \cite{chang2006collaborative} proposed a collaborative technique for face analysis in WCSNs with a dual objective of detecting the camera view closest to a frontal view of the object, and assessing angles between the face directional and all the camera views based on additional fusion of local angle estimates. To gather more information of the stable object, especially face recognition, full-view coverage was introduced in \cite{wang2013achieving} by Wang et al. An object is full-view covered if its face is always a camera to cover it no matter which face direction and the angle between the camera’s viewing direction and the object’s facing direction is less than a predefined parameter $\theta$. The authors proposed a method for full-view coverage verification on a sensing field. After that, they derived an estimation of the sensor density needed for full-view coverage in a random deployment. Based on this work, Wang et al. further studied the problem of constructing a camera barrier in \cite{wang2011barrier}. They proposed a method to select camera sensors from an arbitrary deployment to form a camera barrier and then presented a technique for reducing the number of cameras used since there might be redundant cameras (cameras that can be turned off without breaking the barrier) after barrier is formed. Besides, Ma et al. \cite{ma2012minimum}, proposed a method for constructing camera barrier. With aiming at minimum the number of camera sensors in full-view barrier coverage, the problem is transformed into the shortest path problem from the source to the destination node on graph.

To monitor the object from multiple perspectives, Tseng et al. \cite{tseng2012k} introduced the notion of $k$-angle coverage. To avoid duplicating information from multiple sensors simultaneously monitoring an object, an angle constraint was added, which guaranteed  any two sensors cannot appear in an angle range of $\omega$ around the object (Figure \ref{fig:02}). It was pointed out that if an object is $(k-\omega)$ angle covered, there is no angle larger than $2\pi-(k-1)\omega$ of the object that is not covered by any sensor. This means that an object that is $(k-\omega)$ angle covered is also full-view covered with parameter $\theta=\displaystyle\frac{2\pi-(k-1)\omega}{2}$. Hence, $(k-\omega)$ angle coverage can be considered as a special case of full-view coverage with the number of camera sensors covering the object is fixed. Under this new coverage model, the paper focused on maximizing number of static objects that are covered using minimum number of sensors.
\begin{figure}[!h]
	\centering
	\includegraphics[scale=0.6]{komega.pdf}
	\caption{$O$ is 5-$\omega$-angle covered by $\{S_1, S_2, S_3, S_4, S_5\}$}
	\label{fig:02}
\end{figure}

In \cite{xu2016minimum}, the authors studied the problem of constructing $(k-\omega)$-angle barrier using minimum number of sensors called MkABC. The paper presented MkABC problem in two sensor deployment schemes. Under deterministic deployment, a geometric method was proposed, which used the feature of regular polygon to construct a ($k-\displaystyle\frac{\pi}{k}$)-angle barrier. When sensors are randomly deployed in the ROI, the MkABC becomes more difficult. In this scenario, the authors proposed a grid-based method, where each grid is judged to be $(k-\omega)$-angle covered or not. MkABC problem is then transformed into the shortest path problem on graph. The algorithm used is the same as one used in \cite{ma2012minimum}, which has some problems as aforementioned. Besides, the grid-based method, the downside comes from the size of the grid. The trade-off between grid size, which is directly proportional to the computational cost of the method, and solution accuracy is
a big disadvantage in large-scale WSNs 

Chen et al. \cite{chen2008measuring} first mentioned the problem of measuring the quality of barrier coverage in ODSNs. The authors introduced the notion of $L$-local $k$-barrier coverage to measure the quality of $k$-barrier coverage for a belt region as the maximum value of $L$ that the belt is $L$-local $k$-barrier covered. A belt region is said to be $L$-local $k$-barrier covered if every zone of length $L$ in the region is $k$-barrier covered. The measure always provides the same result when sensor network has already achieved k-barrier coverage, i.e, the probability of detecting the intruder by $k$ sensors is always 100\% which is equivalent to measuring its quality as 1 else zero, is not enough since there might be many different levels of quality coverage of the sensor barrier.  In addition, the considered k-barrier is just combination of consecutive sensing range. Actually, these metrics reflect constructing level of k-barrier, i.e. the closer distance L and the length of the strip region is, the more ROI achieves k-barrier. In contrast, our purpose evaluates quality of collected information in camera sensor barriers. This prompts us to devise a novel metrics called Differential coverage.  

After considering many related works, we see that previous researches about barrier coverage problems in WCSNs are not yet efficient and there are rooms for improvement. Basing on $(k-\omega)$-angle coverage model \cite{xu2016minimum} with some fine-tuned for adapting to monitor mobile objects in barrier coverage in WCSNs, which refers to multiple view coverage model. Therefore, we produce the multiple view barrier coverage problem in WCSNs, then propose method as Dynamic partition to solve this problem. Furthermore, we desire to measure the quality of object's information recorded by sensors network when it crosses the barrier. Since the metrics proposed in \cite{chen2008measuring} is for $k$-barrier coverage model in ODSNs, it cannot apply to our problem. Moreover, this metrics only works when sensors network has not provided $k$-barrier coverage yet. In contrast, we need a metrics for measuring quality coverage of the barrier, which means the barrier must have been already constructed. These have fostered us to devise a new metrics called Differentiation field intensity.




\section{Preliminaries and problem formulation}
\subsection{Preliminaries}

%{\bfseries Definition 1: } {\itshape$(k-\omega)$ coverage}
%\subsubsection{($k-\omega$) region of a k-sensor list}
\begin{df} 
{\itshape multiple-view coverage}
\begin{itemize}
	\item A point $P$ is multiple-view covered with $k$ sensors and angle constraint $\omega$ if there exists a list of $k$ sensors $L = \{S_1, S_2,...,S_k\}$ ordered in counter-clockwise order around $P$, such that $\omega < (\overrightarrow{PS_i}, \overrightarrow{PS_{i+1}}) < \pi, \forall i \in \{1,2,...,k\}$ (consider $k + 1 \equiv 1$)
	\item A region $R$ is said to be multiple-view covered if every point in $R$ is multiple-view covered.
\end{itemize}
Hereinafter, we use two concepts {\itshape multiple-view} coverage and {\itshape ($k,\omega$)} coverage equivalently. With a specific value of $k$ and $\omega$, we always use {\itshape ($k,\omega$)} instead of {\itshape multiple-view} (see figure \ref{fig:02})
\end{df}
\begin{df}
{\itshape Safe region}
\begin{itemize}
	\item Safe region was defined in \cite{tseng2012k}. Figure \ref{saferegion} illustrates the safe region of a line segment $S_1S_2$
	\begin{figure}[h]
		\centering
		\includegraphics[scale=.5]{Hinhanh/safe-region}
		\caption{Safe region of line segment $S_1S_2$}
		\label{saferegion}
	\end{figure}
\end{itemize}
\end{df}
%\begin{verbatim}
%/* some explanation for safe region here */
%\end{verbatim}
\begin{df}
{\itshape Inner safe region}\par
Inner safe region is attached to a side of a polygon, not an arbitrary line segment. Given a polygon {\sc pol}. Let $AB$ is a side of {\sc pol}. The way to determine the inner safe region of $AB$ is as follows:
%\vspace{-5pt}
\begin{itemize}
	\item Choose the midpoint $M$ of $AB$.
	\item Draw an arrow from $M$ toward the inner region of {\sc pol}.
	\item The safe region of $AB$ consists 2 symmetrical parts splitted by line $AB$. The part that the arrow points to is the inner safe region of $AB$.
\end{itemize}
\end{df}
Figure ~\ref{innersafe} is an illustration of inner safe region of $S_1S_2$.
\begin{figure}[!h]
	\begin{center}
	\includegraphics[scale=0.6]{innersafe.pdf}	
	\caption{inner safe region of $S_1S_2$}
	\label{innersafe}
	\end{center}
\end{figure}
\begin{df}
{\itshape($k,\omega$) region of an $k$-sensor list}\\
Given a $k$-sensor list $L = \{S_1,S_2,...,S_k\}$. ($k,\omega$) region of $L$ is the locus of points that are ($k,\omega$) covered by $L$.
\end{df}
\begin{thr}
{\itshape($k,\omega$) region of an $k$-sensor list $L =\{S_1,S_2,...,S_k\}$ is the intersection of inner safe region of every line segment $S_iS_{i+1}$ and the sensing range of all sensors in $L$}.
\end{thr}
\begin{pf}
	Let $\Sigma$ denote the ($k,\omega$) region of $L$. We only consider the case that $\Sigma$ is not empty. Suppose that $P$ is a point inside $\Sigma$, then $P$ is ($k,\omega$) covered by $L$. From definition 1, we have $(\overrightarrow{PS_i}, \overrightarrow{PS_{i+1}}) > \omega$ (1) and $(\overrightarrow{PS_i}, \overrightarrow{PS_{i+1}}) < \pi$ (2),\ $\forall i \in \{1,2,...,k\}$. Condition (1) means that $P$ is inside the safe region of $S_iS_{i+1}$. This safe region has 2 symmetrical parts splitted by line $S_iS_{i+1}$. Condition (2) forces $P$ to be located at only one special part which satisfies $(\overrightarrow{PS_i}, \overrightarrow{PS_{i+1}}) < \pi$. And this part is always the inner safe region of $S_iS_{i+1}$. Hence, $P$ is inside the intersection of inner safe region of every line segment $S_iS_{i+1},\ i=\overline{1,k}$ and the sensing region of all sensors in $L$ (call this intersection $\mho$) (*).\par
\indent On the other hand, if $P$ is inside $\mho$, $P$ satisfies the condition (1) and (2). Thus, $P$ is ($k-\omega$) covered by $L$, which deduce to $P$ is inside $\Sigma$ (**).\par
\indent From (*) and (**), we have $\Sigma \equiv \mho$ and theorem 1 is proved.
\end{pf}
\begin{thr}
{\itshape($k,\omega$) region of an $k$-sensor list $L =\{S_1,S_2,...,S_k\}$ is a convex region}.
\end{thr}
\begin{pf}
The intersection of two convex regions is a convex region. Hence, the intersection of any limited sets of convex region is a convex region. From theorem 1, it is obviously that ($k,\omega$) region of a list $L$ is intersections of convex regions and thus, we have theorem 2 proved.
\end{pf}
Figure \ref{multipleview} is an illustration of this theorem. By that, the shadow area is the ($5,60^o$) region of $L=\{S_1,S_2,S_3,S_4,S_5\}$.
\begin{figure}[h]
\centering
\includegraphics[scale=0.6]{Hinhanh/multipleview-region}
\caption{($5,60^o$) region of $L=\{S_1,S_2,S_3,S_4,S_5\}$}
\label{multipleview}
\end{figure}

We provide some preliminaries used in the following sections of this paper.
\subsubsection{Differentiation Coverage Model}
In the tradition model of attenuation, the algorithm evaluating the coverage from a sensor network toward a point may lead to several exceptional inconsistency, especially in the problem of $k-\omega$ barrier in the sensor network. As a result, in this article we are going to devise a more preferable attenuated model assessing the coverage quality of the sensor network toward a point in the field of interest, the model can later be generalised to evaluate the coverage of the sensor network on a line or a closed region.

Consider a point lying in the sensing range of a certain omnidirectional  or directional sensor. Let us name the sensor $S_i$, the penetration object $P$ with radius $R$ and the considered part $P_\phi$ being positioned at $\phi$ and has length $dl$. Call the distance from $S_i$ to $P_\phi$ as $d_i$, the direction of the sensor compared to the pivot direction as $\phi_i$. Because $R$ is usually inconsiderable compared to $d_i$, we can approximately use $\phi_i$ and $d_i$ as constants with variable $\phi$, the model will be now illustrated as followed.

\begin{figure}[h]
\begin{minipage}{.5\linewidth}
	\centering
	\includegraphics[scale=.75]{groupSensor_Version2.pdf}
\end{minipage}\hfill
\begin{minipage}{.5\linewidth}
	\centering
	\includegraphics[scale=.6]{exposure.pdf}
\end{minipage}
\end{figure}

Firstly, the coverage value of a sensor to a part is directly affected by the distance between the sensor and the part, and the angle at which the part is viewed by the sensor, this results in the formula:
\begin{equation}
\label{eq1}
\mathsf{max}(\frac{A\cos(\phi - \phi_i)}{Rd_i^\lambda}, 0)dl
\end{equation}


with $\frac{A}{R}$ is a constant coefficient, note that the coverage would fall below 0 if the angle between the direction of the part and the direction of the sensor is larger than $\frac{\pi}{2}$, so we need to set it to 0 in that case. Rewrite $dl = Rd\phi$, we have:
\begin{equation}
\label{eq2}
\mathsf{max}\left\{\frac{A\cos(\phi - \phi_i)}{d_i^\lambda},\ 0\right\}d\phi
\end{equation}

However, it is obviously unnecessary to obtain too much detailed information from the object in the sensing field. This leads to the existence of a constant $E_{\mathsf{max}}$ which corresponds to the maximum necessary coverage on a part with unit length of the circle. To isolate the value from the relative constant $A$, we rewrite it to the Minimum sensing radius $E_{\mathsf{max}} = \frac{A}{d_{\mathsf{min}}^\lambda}$. As a result, our coverage formula could be rewritten as:
\begin{equation}
\label{eq3}
\mathsf{max}\left\{0,\ \mathsf{min}\Big\{\frac{A}{d_{\mathsf{min}}^\lambda}, \frac{Acos(\phi - \phi_1)}{d_i^\lambda}\Big\}\right\}d\phi
\end{equation}

As a result, the coverage on a part $P_\phi$ of several sensors $S_i$, as illustrated above, is the maximum of the coverage on that part of every covered sensor:

\begin{equation}
\label{eq4}
E_\phi(P) = \mathsf{max}\left\{0,\ \mathsf{min}\Big\{\frac{A}{d_{\mathsf{min}}^\lambda},\ \underset{S_i}{\mathsf{max}}\big\{\frac{A\cos(\phi - \phi_i)}{d_i^\lambda}\big\}\Big\}\right\}d\phi
\end{equation}


In short, the coverage on a part of a certain set of sensors is calculated from the largest value of $\displaystyle\frac{Acos(\phi - \phi_1)}{d_i^\lambda}d\phi$ across all sensors, the result then will be a value in the close interval $[0, E_{\mathsf{max}}d\phi]$ that closest to the above computed value. The total coverage on the intruder is the sum of the coverage on its small parts. Combined with the differentiated form of the formula above, the total coverage would be the integral on all of its parts. As a result, we receive the formula for the total coverage on the intruder at a certain point in the sensing field:

$$E(P) = \uint\limits_0^{2\pi}{\mathsf{max}\left\{0,\ \mathsf{min}\Big\{\frac{A}{d_{\mathsf{min}}^\lambda}, \ \underset{S_i}{\mathsf{max}}\big(\frac{A\cos(\phi - \phi_i)}{d_i^\lambda}\big)\Big\}\right\}d\phi}$$

In conclusion, a new model of coverage is devised which may prove to be exceptionally effective in measuring the coverage efficiency of sensor networks in not only the tradition coverage problem but also in more complex ones such as the problem of $full\hspace{1mm}view$ or $k-\omega$ coverage. The new model is proposed with detailed and precise logical progress, successfully adapts the strong points of both the All-Sensor Field Intensity and the Closest-Sensor Field Intensity model, handling preferably the cooperation of multiple sensors in the network without overrating the repetition of captured information.

\subsubsection{Coverage of a barrier}
\label{barrier}

\begin{df}{\itshape($k-\omega$) barrier}\\
	A ($k-\omega$) barrier $B$ is a connected region from the left side to the right side of the monitoring region and satisfies that $B$ is ($k-\omega$) covered.
\end{df}
\begin{df}{\itshape($k-\omega$) barrier coverage}\\
	A region achieves ($k-\omega$) barrier coverage if there exists a ($k-\omega$) barrier in that region.\par
\end{df}

A $k-\omega$ barrier (a barrier) is a region connects the left and right side of the sensing field in which all the points are $k-\omega$ covered. Typically, a barrier is fairly narrow, and penetration objects usually intersect the barrier only at a small part on their paths. As a result, a proper metric to assess the efficiency of the barrier would be the coverage density of it.

With the same set of sensors considered, in the range of coverage of all elements of that set, the coverage function is always continuous. Since a barrier is consisted of several separated parts each of which is $k-\omega$ covered by a common set of sensors, the coverage density of the barrier can be defined as the quotient of the total coverage in the barrier and the area of that area, with the total coverage being formulated as the integral of the coverage function over the barrier region. Call the barrier region $B$ with area $S_B$, the coverage density over $B$, which is $D_B$ can be formulated as
\begin{equation}
\label{eqE1}
E(B) = \iint_B{E(x, y)dxdy}.\frac{1}{S_B}
\end{equation}


\subsection{Problem formulation}
\subsubsection{Verify the ($k-\omega$) barrier coverage}
The problem is formulated as follows. Given a set of $n$ sensors $S=\{S_1,S_2,...,S_n\}$ and a rectangular region $\Omega$ with the length of $L$ and the width of $W$. $\Omega$ is called the monitoring region and camera sensors in $S$ are deployed according to uniform deployment scheme in $\Omega$ to serve the purpose of observation in $\Omega$. The uniform deployment scheme means that total $ n $ sensors are deployed randomly, uniformly and independently.\par

The objective of the problem is to verify if the monitoring region $\Omega$ achieves ($k-\omega$) barrier coverage. In other words, we need to determine if there exists a ($k-\omega$) barrier $B$ in the monitoring region $\Omega$. If there is none, $\Omega$ will not guarantee security requirements and the sensors need to be re-deployed.\par

Unlike omni-directional sensor, which only provides information about detection of the object, camera sensor is typically directional sensor and can be used to obtain multimedia information of the object. Each camera sensor can be denoted by a 4-tuple $\{S_i, R, \alpha, \varphi_i\}$, where $S_i$ is the location of sensor $i$, $R$ is the sensing radius and $\alpha$ is half of the sensing angle. We assume that all sensors have the same sensing radius and sensing angle. In reality, sensing range of camera sensor is usually less than $\pi$, so we also have an assumption that $\alpha < \displaystyle\frac{\pi}{2}$. The last parameter of a camera sensor, $\varphi_i$, is the facing direction of sensor $i$, which is uniformly distributed in $[0,2\pi]$ \par
Figure \ref{sensor} shows information of sensor $S_i$.
%\begin{figure}[h]
\begin{center}
	\begin{tikzpicture}[line cap=round,line join=round,>=triangle 45,x=1.0cm,y=1.0cm, scale=0.7]
	\clip(-1.4631724023603414,-1.2270677911720365) rectangle (9.705105880506954,7.2715025356313285);
	\draw [shift={(-0.01708565450953481,-0.034171309019069396)},line width=0.8pt]  plot[domain=0.39431447531259756:1.4415120265091952,variable=\t]({1.*6.455385348683685*cos(\t r)+0.*6.455385348683685*sin(\t r)},{0.*6.455385348683685*cos(\t r)+1.*6.455385348683685*sin(\t r)});
	\draw [line width=0.8pt] (0.8151713441050572,6.367340097536184)-- (-0.01708565450953481,-0.034171309019069396);
	\draw [line width=0.8pt] (-0.01708565450953481,-0.034171309019069396)-- (5.942914345490465,2.445828690980931);
	\draw [->,line width=0.8pt] (-0.01708565450953481,-0.034171309019069396)-- (4.9061183101524755,6.403378415301186);
	\draw [shift={(0.34228704361157486,0.2425085445513223)},line width=0.8pt]  plot[domain=0.10490095301338:1.2073267732101145,variable=\t]({1.*0.41131528053771554*cos(\t r)+0.*0.41131528053771554*sin(\t r)},{0.*0.41131528053771554*cos(\t r)+1.*0.41131528053771554*sin(\t r)});
	\draw [shift={(0.1558003047047065,0.38512682194880327)},line width=0.8pt]  plot[domain=0.6284997286116789:1.7309255488084125,variable=\t]({1.*0.4113152805377156*cos(\t r)+0.*0.4113152805377156*sin(\t r)},{0.*0.4113152805377156*cos(\t r)+1.*0.4113152805377156*sin(\t r)});
	\draw [->,shift={(0.9176828234963075,0.44900218753382565)},line width=0.8pt]  plot[domain=-0.5200419291940417:1.4379551801049384,variable=\t]({1.*0.9035766884728488*cos(\t r)+0.*0.9035766884728488*sin(\t r)},{0.*0.9035766884728488*cos(\t r)+1.*0.9035766884728488*sin(\t r)});
	\draw [->,line width=0.8pt] (-0.017085654509534805,-0.03417130901906939)-- (9.,0.);
	\draw [<-,line width=0.8pt] (0.3657817242610962,6.425764996355688)-- (0.06760477955619315,4.132262970085357);
	\draw [line width=0.8pt,->] (-0.16829832964859293,2.317755616070765)-- (-0.46647527435349595,0.024253589800434013);
	%\draw [->,line width=0.8pt] (0,0)-- (9.,0.);
	
	%\node[draw] (first) at (4,4){1};
	\begin{normalsize}
	\node at (0.95, 0.68) {$\alpha$};
	\node at (0.42, 1.09) {$\alpha$};
	\node at (-0.3,-0.3) {$S_i$};
	\node at (9.,-0.3) {$x$};
	\node at (1.81, 1.29) {$\varphi_i$};
	\node [rotate=90] at (-0.05,3.23) {$R$};
	\end{normalsize}
	\end{tikzpicture}
	\caption{Example of a camera sensor}
	\label{sensor}
\end{center}
\end{figure}

\noindent The input and output of the problem are followings:\\[7pt]
{\bfseries Input}
\begin{itemize}
	\item $L, W$: Length and width of the monitoring region $\Omega$ respectively.
	\item $n$: Number of camera sensors.
	\item $S = \{S_1,S_2,...,S_n\}$: Set of camera sensors. $S_i$ denotes the $i$-th camera sensor and also denotes the location of that sensor.
	\item $R$: Radius of camera sensors.
	\item $\alpha$: Half of the sensing angle of camera sensors.
	\item $\varphi_i$: Orientation angle view of $S_i$ where $i=\overline{1,n}$.
	\item $k, \omega$ : The conditional parameter of the problem.
\end{itemize}
{\bfseries Output}
\begin{itemize}
	\item The yes/no answer that the monitoring region achieves ($k-\omega$) barrier coverage. 
\end{itemize}

\subsubsection{Evaluate the quality of a ($k-\omega$) barrier}
The problem is formulated as follows. Given a barrier $B$ in a sensing field containing several connected regions $B_i$ from the left to the right edge of the field. Each $B_i$ is a closing field that is ($k-\omega$) covered by a $k$-list of sensors $P_i$.

The objective of the problem is to find the coverage of the barrier regarding our devised metric. The process is to assess the quality of the found barrier and compare the result with other settings of parameters to analyze the effect of each parameter to the quality of the sensing field and find the best combination of settings to achieve our desire.

\noindent The input and output of the problem are followings:\\[7pt]
{\bfseries Input}
\begin{itemize}
	\item $\{B_i\}$: The set of closing region connect the left and the right edge of the sensing field.
	\item $\{P_i\}$: The set of $k$-list of sensors, the $P_i$ is known to ($k-\omega$) cover the region $B_i$.
\end{itemize}
{\bfseries Output}
\begin{itemize}
	\item The coverage value of the ($k-\omega$) barrier.
\end{itemize}

\section{Proposed algorithm}
\subsection{Verify the ($k-\omega$) barrier cover}
To solve this problem, the monitoring region is partitioned into several small rectangles using the proposed Dynamic Partition method. After that, we try to figure out if there is a continuous barrier from the left side to the right side of the region consisting of rectangles which are ($k-\omega$) covered. The details are shown in the subsequent sections.

\subsubsection{$(k-\omega)$ verification on a rectangle}
% introduction idea
Based on theorem 2, we can conclude that in order to verify the $(k-\omega)$ coverage on a rectangle, it is only necessary to check whether all four vertices of that rectangle is $(k-\omega)$ covered by an common list of sensors. Using this conclusion, we propose an algorithm to verify whether a rectangle is $(k-\omega)$ covered or not, and for the optimisation of the second problem, we try to find the list that $(k-\omega)$ cover the rectangle with the largest coverage value toward it. 
% 4 steps
The idea of our algorithm can be described as follows:
\begin{description}
	\item[Step 1:] First, we find a set of sensors $G$ that cover four vertices of the rectangle $ ABCD $.
	\item[Step 2:] Choose a list of $k$ sensors from $G$ satisfies that the point $A$ is $(k-\omega)$ covered by these $k$ sensors.
	\item[Step 3:] If found any, check all of those lists and take the ones that also $(k-\omega)$ cover the 3 remaining vertices of the considered rectangle.
	\item[Step 4:] Among all the list taken, return the one with the largest coverage value toward the considered rectangle.
\end{description}
The key to implement this idea is at step 2. Our approach to this problem is very natural. First, sort $G$ in counter-clockwise order around $A$. Then, we consider each sensor in $G$ sequentially. If the sensor being considered satisfies some conditions, we put it into a list (call this list $L$). We do that until size of $L$ is equal to $k$. Then, $L$ is called a valid list. Figure \ref{finding} illustrates how to choose a valid list. In figure \ref{finding}, black vector denotes the sensor that is chosen to put into the list, while red vector denotes candidates to be chosen.

\begin{figure}[h]
	\begin{minipage}{.3\linewidth}
		\includegraphics[scale=.5]{setSensors_1.pdf}
	\end{minipage}
	\hfill
	\begin{minipage}{.3\linewidth}
		\includegraphics[scale=.5]{setSensors_2.pdf}
	\end{minipage}
	\hfill
	\begin{minipage}{.3\linewidth}
		\includegraphics[scale=.5]{setSensors_3.pdf}
	\end{minipage}
	\\
	\begin{minipage}{.3\linewidth}
		\includegraphics[scale=.5]{setSensors_4.pdf}
	\end{minipage}
	\hfill
	\begin{minipage}{.3\linewidth}
		\includegraphics[scale=.5]{setSensors_5.pdf}
	\end{minipage}
	\hfill
	\begin{minipage}{.3\linewidth}
		\includegraphics[scale=.5]{setSensors_6.pdf}
	\end{minipage}\\
	\caption{$L=\{S_1, S_4, S_9, S_{13}, S_{17}\}$ is a valid list with $k=5, \omega=60^o$}
	\label{finding}
\end{figure}

As aforementioned, when considering a sensor in $G$, it must satisfies some conditions to become candidate to be put into the list $L$. Suppose that at some point of the finding process, the list has {\sf\itshape index} elements and {\sf\itshape $L$[index] = $G$[cur]}, {\sf\itshape 1} $\leq$ {\sf\itshape cur} $\leq$ {\sf\itshape $n$}, $n$ is size of $G$. If {\sf\itshape $G$[next]} is chosen to be the next element in $L$, it must satisfy two conditions:
\begin{itemize}
	\item $(\overrightarrow{\vphantom{\overrightarrow{P}}PG\textsf{[cur]}}, \overrightarrow{\vphantom{\overrightarrow{P}}PG\textsf{[next]}}) > \omega \qquad (1)$
	\item $(\overrightarrow{\vphantom{\overrightarrow{P}}PG\textsf{[next]}}, \overrightarrow{\vphantom{\overrightarrow{P}}PL\textsf{[1]}}) > (k-\textsf{index})\omega \qquad (2)$
\end{itemize}
From definition of ($k-\omega$) coverage, condition (1) is clearly necessary. However, it's not sufficient for {\sf\itshape $G$[next]} to become a candidate for the next position in $L$. If $L$ is a valid list, we have $(\overrightarrow{\vphantom{\overrightarrow{P}}PL[i]}, \overrightarrow{\vphantom{\overrightarrow{P}}PL[i+1]}) > \omega, i = \overline{1,k}$ (consider $k + 1 = 1$). Hence, $(\overrightarrow{\vphantom{\overrightarrow{P}}PL\textsf{[index + 1]}}, \overrightarrow{\vphantom{\overrightarrow{P}}PL[1]}) = \displaystyle\sum\limits_{i=\textsf{\scriptsize index} + 1}^{k}(\overrightarrow{\vphantom{\overrightarrow{P}}PL[i]}, \overrightarrow{\vphantom{\overrightarrow{P}}PL[i+1]}) > (k-\textsf{index})\omega$.
Since we are choosing candidate for {\sf (index + 1)-th} element in $L$, {\sf $G$[next]} corresponds to {\sf $L$[index + 1]}. Thus, (2) is also a necessary condition.\\[7pt]
Algorithm \ref{algo00} and Algorithm \ref{algo01} shows the details of our algorithm. The function in Algorithm \ref{algo01} is a support function for Algorithm \ref{algo00}. {\tt RecurFinding(cur)} is a recursive function that finds candidate for {\sf (index + 1)-th} position in $L$ knowing that there is a set {\sf cur} containing the chosen sensors, in which the last sensor has index {\sf last}. It considers elements in $G$ sequentially from {\sf (last + 1)-th} element and checks if these elements satisfy condition (1) and (2). The return value of {\tt RecurFinding(cur)} is a set containing all the lists of sensors that ($k-\omega$) cover the point $P$ which takes the current sublist ($\{L\textsf{[1]}, L\textsf{[1]},..., L\textsf{[index]}\}$) as its first {\sf index} elements exists. This return value is used to support the recursion process of the algorithm.

\noindent\begin{minipage}{.49\linewidth}
	\begin{algorithm}[H]
		\DontPrintSemicolon 
		\SetAlgoLined
		\newcommand{\forcondi}[2]{\ensuremath{#1 \in #2}}
		\SetKwData{found}{found}
		\SetKwData{Gsize}{G.size}
		\SetKw{true}{true}
		\SetKw{false}{false}
		\SetKwData{and}{\&\&}
		\SetKwData{equal}{==}
		\SetKwData{greater}{$\geq$}
		\SetKwData{notFound}{!found}
		\SetKwData{k}{k}
		\SetKwData{n}{n}
		\newcommand{\forcond}[3]{\ensuremath{#1 = #2\ \KwTo\ #3}}
		\BlankLine
		\KwIn{
			A point {\sf P} and a set {\sf G} consisting of {\sf n} sensors that cover {\sf P}.
		}
		\KwOut{
			\k sensors that ($\mathsf{k-\omega}$) covers {\sf P}. There is possibility that no output is found.
		}
		\BlankLine
		\BlankLine
		Let {\sf L} store the output\;
		Sort {\sf G} in counter-clockwise order around {\sf P}\;
		$L \leftarrow \emptyset$\;
		\For{\forcond{i}{1}{n}}{
			$temp \leftarrow {\tt RecurFinding(G{i})}$\;
			$L \leftarrow L \cup temp$\;
		}
		\caption{Find all lists of $k$ sensors that ({\sf $k-\omega$}) covers point $P$}
		\label{algo00}
	\end{algorithm}
\end{minipage}
\hfill
\begin{minipage}{.49\linewidth}
	\begin{algorithm}[H]
		\DontPrintSemicolon
		%	\SetAlgoNoLine
		%	\SetAlgoNoEnd
		\newcommand{\forcondi}[2]{\ensuremath{#1 \in #2}}
		\SetKwData{found}{found}
		\SetKwData{Gsize}{G.size}
		\SetKw{true}{true}
		\SetKw{false}{false}
		\SetKwData{and}{\&\&}
		\SetKwData{equal}{==}
		\SetKwData{greater}{$\geq$}
		\SetKwData{notFound}{!found}
		\SetKwData{k}{k}
		\SetKwData{n}{n}
		\SetKwData{cur}{cur}
		\SetKwArray{G}{G}
		\SetKwArray{L}{L}
		\SetKwData{P}{P}
		\SetKwFunction{Recur}{RecurFinding}
		\SetKwData{paratwo}{\tt index + 2, i}
		\SetKwData{paraone}{index + 1}
		\SetKwData{index}{index}
		\SetKw{return}{return}
		\SetKwProg{Fn}{}{}{end}
		\newcommand{\forcond}[3]{\ensuremath{#1 = #2\ \KwTo\ #3}}
		\BlankLine
		\KwIn{
			A list of sensors that is currently chosen, contains \index sensors.
		}
		\KwOut{
			All lists of \k sensors that ($k-\omega$) cover the point \sf P beginning with the input list.
		}
		\BlankLine
		\BlankLine
		\Fn{\tt RecurFinding({\sf cur})}{
			$L \leftarrow \emptyset$\;
			$\index \leftarrow$ cardinality of {\sf cur}\;
			\If{m \equal k}{\return {\sf \{cur\}}}
			$last \leftarrow$ index of last element in {\sf cur}\; 
			$first \leftarrow$ index of first element in {\sf cur}\;
			\For{\forcond{i}{last + 1}{\n}}{
				\If{
					$(\overrightarrow{\vphantom{\overrightarrow{P}} \P\G{last}}, \overrightarrow{\vphantom{\overrightarrow{P}} \P\G{i}}) > \omega$ \and
					$(\overrightarrow{\vphantom{\overrightarrow{P}} \P\G{i}}, \overrightarrow{\vphantom{\overrightarrow{P}} \P\G{first}}) > (\k - \index + 1)\omega$
				}{
					$temp \leftarrow$ {\tt RecurFinding({\sf cur} + (\G{i}))}\;
					$L \leftarrow L \cup temp$\;
				}
			}
			\return L\;
		}
		\caption{Find {\sf index + 1} element in {\sf L}}
		\label{algo01}
	\end{algorithm}
\end{minipage}

%\begin{figure*}[!h]
%\noindent\begin{minipage}{.49\linewidth}
%\includegraphics{Algo_1.pdf}
%\end{minipage}\hfill
%\begin{minipage}{.49\linewidth}
%\includegraphics{Algo_2.pdf}
%\end{minipage}
%\end{figure*}

The first element of $L$ can be any sensor in $G$ since it doesn't require any condition. For convenient, we choose $L[1]=G[1]$. Thus, {\tt RecurFinding(\{G[1]\})} is called to start the finding process. After function call {\tt RecurFinding(\{G[1]\})}, the function will return all the satisfied lists containing G[1]. The finding process stops when we have called {\tt RecurFinding(\{G[$i$]\})} with every $i$ from 1 to n. And the algorithm will output a set containing all the lists of sensors that ($k-\omega$) cover the considered point $P$.

\subsubsection{Finding a barrier in a monitoring region}


a, Partitioning the monitoring region by Dynamic Partition method

\label{subsection1}

To find a barrier in the monitoring region, we first determine the areas that are $(k-\omega)$ covered inside the monitoring region. To solve this problem, we partition the monitoring region into multiple small rectangles and check whether these rectangles are $(k-\omega)$ covered or not. However, uniform partitioning often requires a high computation time especially when the monitoring region is large. To overcome this challenge, we propose a new partition method called Dynamic Partition. The idea is: only the rectangle regions which are not $(k-\omega)$ covered will be partitioned into smaller rectangles, otherwise, they are kept untouched. \\
The first rectangle to be checked is the monitoring region. 
Using the algorithm in subsection 1, if a rectangle is $(k-\omega)$ covered, mark it as true, otherwise, split it into four equal sub-rectangles. After a rectangle is split, smaller rectangles are generated and the process of checking and splitting is applied to these new rectangles. A rectangle will not be split if it is $(k-\omega)$ covered or its size reaches a predefined limited value. The smaller the limited size is, the more precise the result of our algorithm gets. This condition guarantees our algorithm not to go into an infinite loop. The process is illustrated in Figure \ref{dynamic}.
%\input{DynamicPartition}
\begin{figure}[h]
	\begin{center}
		\includegraphics[scale=1.]{Dynamic_Partition.pdf}
	\end{center}
	\caption{Illustration of Dynamic Partitioning}
	\label{dynamic}
\end{figure}

The pseudo code of the method is described in Algorithm \ref{alg1}
%

%
\begin{center}
	\begin{minipage}{.7\linewidth}
		\begin{algorithm}[H]
			\DontPrintSemicolon
			\SetAlgoLined
			\newcommand{\forcondi}[2]{\ensuremath{#1 \in #2}}
			\SetKwData{Rcen}{root.center}
			\SetKwData{root}{rootRec}
			\SetKwData{RX}{root.sizeX}\SetKwData{RY}{root.sizeY}
			\SetKwData{L}{L}\SetKwData{W}{W}
			\SetKwData{QUEUE}{Q}
			\SetKwData{tempNode}{tempRec}
			\SetKwData{coveredGroup}{coveredGroup}
			\SetKwFunction{FAC}{findAndCheck}
			\SetKwData{coveredNodes}{coveredRectangles}
			\SetKwData{DMAX}{DMAX}
			\SetKwFunction{split}{split()}
			\SetKwFunction{makeNode}{makeNode}
			\SetKwFunction{dequeue}{dequeue}
			\SetKwFunction{check}{check}
			\SetKwData{tnRank}{tempRec.rank}
			\SetKwData{element}{element}
			\BlankLine
			\KwIn{
				\begin{itemize}
					\item Length and width of the monitering region $\Omega$: \L, \W respectively
					\item A set of sensors $S$
					\item Maximum depth of quad-tree: \DMAX
				\end{itemize}
			}
			\KwOut{A set of covered rectangles}
			\BlankLine
			\BlankLine
			Let \coveredNodes store the output\;
			Let \root denote the monitoring region\;
			$\QUEUE\gets\emptyset$\;
			add \root to \QUEUE \;
			\While{$\QUEUE \neq\emptyset$}{
				$\tempNode\gets$ \dequeue{\QUEUE}\;
				Check if \tempNode is $(k-\omega)$ covered\;
				\uIf{\tempNode is $(k-\omega)$ covered}{add \tempNode to \coveredNodes\;}
				\ElseIf{\tnRank $<\DMAX$}{split \tempNode in into 4 sub-rectangles\;
					add 4 sub-rectangles of \tempNode to \QUEUE}
			}
			\caption{Dynamic Partition}
			\label{alg1}
		\end{algorithm}
	\end{minipage}
\end{center}

b, Finding a $(k-\omega)$ coverage barrier

After procedure in \ref{subsection1}, we now have a set $ R_{cover} $ of rectangles that are $(k-\omega)$ covered. To find a $(k-\omega)$ coverage barrier, we need to find a continuous area formed from rectangles in $R_{cover}$ that connects the left side to the right side of $\Omega$. The method is to transform the rectangles set into a graph. Each vertex in the graph corresponds to a rectangle in $ R_{cover} $. Two vertices are considered adjacent if the corresponding rectangles share at least one point. Two virtual vertices are added to the graph, source vertex $ s $ and sink vertex $ t $. All vertices corresponding to the rectangles lying on the left side of $\Omega$ are adjacent to $s$ and all vertices corresponding to the rectangles lying on the right side of $\Omega$ are adjacent to $t$. After the graph is constructed, we use Breath First Search algorithm to find a path from $s$ to $t$. If a path is found, we conclude that there exists a $(k-\omega)$ barrier in the monitoring region. Otherwise, the barrier does not exist.

\subsection{Evaluate the quality of a ($k-\omega$) barrier}
\label{baCal}
To assess quality of coverage of achieve sensor barrier, the coverage model implements the idea of Divide-And-Conquer. The object is differentiated into several small parts. Each of them is then evaluated separately, then added together to obtain the total coverage of the sensor network on the intruder at the considered position. Each small part of the circle, in turn, is calculated with each sensor that cover it, then the largest coverage value is taken. This operation will prevent the coverage overrated from several sensors having similar position toward the considered point, which in real life will serve no purpose of obtaining more information of the specific part on the intruder. 

The algorithm takes the nodes forming a barrier and the $k$-list of sensors associating with each node as the input and compute the coverage on the input barrier.

The coverage of the barrier is calculated as the average of every node which forms that barrier with the weight assigned as the area of each node. With $B_i$ as the nodes forming the barrier $B$, we have

$$E(B) = \iint_{B}E(P)dxdy.\frac{1}{S_B}$$

$$= (\sum_i(\iint_{B_i}E(P)dxdy)).\frac{1}{S_B}$$

$$=(\sum_iE(B_i).S_{B_i}).\frac{1}{S_B} $$

$$=(\sum_iE(B_i).S_{B_i}).\frac{1}{\sum_iS_{B_i}}$$

As a result, this calculation method is consistent with our definition of coverage on the barrier in the \ref{barrier} section, hence may provide preferable assessment on each setting of parameters.

Since it is impossible and unnecessary to compute the exact coverage value of each note, it is sufficient to publish a method to estimate an approximation of the coverage on the considered node. Take into account the fact that in each node is ($k-\omega$) covered by an unique $k$-list of sensors, hence the coverage value inside the node is a continuous function. As a result, we can create a dense grid in each node, and estimate the node coverage with the average of the vertices on the discrete grid. For convenience, the size of the grid is fixed to be the size of the node which will not be split further in \ref{subsection1}.

\section{Experimental results}

\subsection{Simulation method}

This part will analyze the effect of several parameters on 3 aspects of the result, which is the probability of creating barrier, the average exposure along the multiple view barrier coverage and the overall computational time. The algorithm is performed on every instance and keep recording 3 data, the creation of barrier, the computation time and the exposure on a barrier if there is one. Then, the result are combined for all instances of the same parameter settings to achieve the probability of barrier creation, the average computational time and the average exposure value on the found barriers.

\subsection{System setting and parameters setting}
%Đoạn này hư cấu sửa sau :v
System settings

All the experiments are performed on a personal computer with core Intel Core i7-7700HQ, 8GB of DDR4 RAM running on Windows 10 Home, the programming language used to simulate the algorithm is Java 11.

Parameter settings

The sensing fields in all experiments are presented as rectangles with the size of 200m x 50m. Sensor nodes are deployed uniformly randomly in a rectangle with each side extended compared to the sides of the sensing field a distance equal to the sensing radius of each sensor in order to guarantee the uniform distribution regarding sensing area inside the sensing field. Each set of parameters contains several independent random topologies to conduct the algorithm on and measure the target indexes. Furthermore, each instance of experiment is conducted with both node handling methods. Altogether there are 42000 experiments on 210 instances of parameters which were analyzed with our algorithm. The details are as follows:
\begin{table}[h!]
	\centering
	\begin{tabular}{l | r}
		Length & 200 \\
		Width & 50 \\
		Sensing Radius & 30 \\
		Minimum sensing radius & 5 \\
		Sensing angle & 90 \\
		k & 3, 4, 5, 6 \\
		The number of topologies for each instance & 100 \\
		Node handling method & Max, Random
		
	\end{tabular}
	\caption{General parameters}
\end{table}

\begin{table}[h!]
	\centering
	\begin{tabular}{l | c | c | c | c}
		k & 3 & 4 & 5 & 6\\
		$\omega$ & 90 - 115 & 55 - 80 & 40 - 60 & 35 - 50\\
	\end{tabular}
	\caption{Problem parameters}
\end{table}

\subsection{Computation results}

Obviously, the max method will offer a greater coverage value, while sacrifice some speed to choose the appropriate list in each situation, this result will be illustrated clearly in the all of the following figures in this section.

\subsubsection{Effect of $\omega$ on algorithm performance}

$\omega$ is an important parameter in the $k-\omega$ coverage model. As a result, this parameter has a considerable impact on the output of the algorithm. Because $\omega$ is a lower bound for the angle between two consecutive sensors in the perspective of the considered point, every sensor set that satisfies the condition with large $\omega$ would also successfully make a $k-\omega$ cover with lower $\omega$. In short, a decrease in parameter $\omega$ may result in an expansion in the result space of the algorithm. This leads to two different consequences. On the one hand, there would be more sets of sensor $k-\omega$ cover a single node, which means that the exposure of that node is likely to be lifted. However, on the other hand, the lower value of $\omega$ could reduce the average rank of the covered nodes, as the nodes are more easily covered, which leads to a lower exposure, since the sets that cover the bigger node tend to position further than the sets covering the smaller ones.

As a consequence, firstly, with a lower value of $\omega$, the algorithm would offer a greater chance of $(k-\omega)$ barrier existence, and probably also a greater exposure on the obtained barriers. However, the probability of forming barriers can never exceed $100\%$, the curve that represents the barrier probability would approach $100\%$ and does not rise higher with lower $\omega$. Furthermore, a high exposure usually occurs when the node is exposed at every direction, which may satisfy the condition with high $\omega$. Consequently, the exposure value would eventually approach a bound when the value of $\omega$ decrease.

On the other hand, a lower value of $\omega$ results in a larger searching space, which leads to a drastic rise in computation time. As a result, it is suitable to choose a sufficient $\omega$ so that the barrier probability and the barrier exposure approach its upper bound while the computation time is still acceptable.

\begin{figure}[h]
	\begin{subfigure}[t]{.5\textwidth}
		\centering
		\includegraphics[scale=.8]{Hinhanh/OmegaEffect/coverage/k4.pdf}
		\caption{k = 3}
	\end{subfigure}
%	\hfill
	\begin{subfigure}[t]{.5\textwidth}
		\centering
		\includegraphics[scale=.8]{Hinhanh/OmegaEffect/coverage/k3.pdf}		
		\caption{k = 4}
	\end{subfigure}
\caption{Effect of $\omega$ on the average coverage value of the found barrier with $k = 3$ and $k = 4$}
\label{fig:}
\end{figure}
%
\begin{figure}[h]
	\begin{subfigure}[t]{.5\textwidth}
		\centering
		\includegraphics[scale=.8]{Hinhanh/OmegaEffect/time/k3.pdf}		
		\caption{k = 3}
	\end{subfigure}
	\begin{subfigure}[t]{.5\textwidth}
		\centering
		\includegraphics[scale=.8]{Hinhanh/OmegaEffect/time/k4.pdf}		
		\caption{k = 4}
	\end{subfigure}
\caption{Effect of $\omega$ on the average computational time in $ms$ of the algorithm with $k = 3$ and $k = 4$}
\label{fig:}
\end{figure}
%
\begin{figure}[h]
	\begin{subfigure}[t]{.5\textwidth}
		\centering
		\includegraphics[scale=.8]{Hinhanh/OmegaEffect/probability/k3.pdf}		
		\caption{k = 3}
	\end{subfigure}
	\begin{subfigure}[t]{.5\textwidth}
		\centering
		\includegraphics[scale=.8]{Hinhanh/OmegaEffect/probability/k3.pdf}		
		\caption{k = 4}
	\end{subfigure}
\caption{Effect of $\omega$ on the probability of existence of multiple view barrier with $k = 3$ and $k = 4$}
\label{fig:}
\end{figure}

\subsubsection{Effect of sensor number on algorithm performance}

Like the effect of $\omega$, a larger value of sensor number would lead to a larger searching space. However, in this occasion, the negative effect on exposure seems to be more important. As a result, the barrier exposure tends to fall slowly as the sensor number rises. Furthermore, the large number of sensors leads to a huge computational work. This results in the computation time surge dramatically as the sensor number grows. Thus, it is suitable to choose a sufficient sensor number so that the barrier probability approaches $100\%$, while the barrier exposure has not fallen too considerable and the computation time is still acceptable.

\begin{figure}[h]
	\begin{subfigure}{.5\textwidth}
		\centering
		\includegraphics[scale=.8]{Hinhanh/SensorNumberEffect/coverage/k3omega105.pdf}
		\caption{k = 3, $\omega$ = 105}
	\end{subfigure}
	\begin{subfigure}{.5\textwidth}
		\centering
		\includegraphics[scale=.8]{Hinhanh/SensorNumberEffect/coverage/k4omega65.pdf}
		\caption{k = 4, $\omega$ = 65}
	\end{subfigure}
\caption{Effect of sensor number on the average coverage value of the found barrier with some typical values of $k$ and $\omega$}
\label{fig:}
\end{figure}
%
\begin{figure}[h]
	\begin{subfigure}{.5\textwidth}
		\centering
		\includegraphics[scale=.8]{Hinhanh/SensorNumberEffect/time/k3omega105.pdf}
		\caption{k = 3, $\omega$ = 105}
	\end{subfigure}
	\begin{subfigure}{.5\textwidth}
		\centering
		\includegraphics[scale=.8]{Hinhanh/SensorNumberEffect/time/k4omega65.pdf}
		\caption{k = 4, $\omega$ = 65}
	\end{subfigure}
\caption{Effect of sensor number on the average computational time in $ms$ of the algorithm with some typical values of $k$ and $\omega$}
\label{fig:}
\end{figure}
%
\begin{figure}[h]
	\begin{subfigure}{.5\textwidth}
		\centering
		\includegraphics[scale=.8]{Hinhanh/SensorNumberEffect/probability/k3omega105.pdf}
		\caption{k = 3, $\omega$ = 105}
	\end{subfigure}
	\begin{subfigure}{.5\textwidth}
		\centering
		\includegraphics[scale=.8]{Hinhanh/SensorNumberEffect/probability/k4omega65.pdf}
		\caption{k = 4, $\omega$ = 65}
	\end{subfigure}
\caption{Effect of sensor number on the probability of existence of multiple view barrier with some typical values of $k$ and $\omega$}
\label{fig:}
\end{figure}

\subsubsection{Effect of $k$ on algorithm performance}

The parameter$k$ is affected the achieved results the most regarding all 3 aspects. This is because the change in k would manipulate the problem entirely, an answer with a value of $k$ would not be an answer with another value of $k$. As a consequence, the achieved results are drastically different among every value of $k$.

As mentioned in previous parts, generally, the exposure of the barriers would not be much different from the others. As a result, we may reach a conclusion that for every value of $k$, it is possible to define a critical value of barrier exposure which denotes the largest achieved value of exposure for a certain value of $k$. And this critical exposure value could be use to compare the performance of the problem with different values of $k$.

Regarding this metric, in general, as there are more sensors that cover a certain point, an increase in the value of $k$ may lead to a larger critical exposure. However, since the function of $cos(x)$ has a derivative getting lower as the value of $x$ comes close to 0, and the effect of increasing $k$ on decreasing the sight angle of sensors to the parts of the intruder ($\phi - \phi_i$) may reduce with larger $k$. As a result, the critical exposure value will eventually reach a bound when the value of $k$ keep climbing.

Finally, the effect of $k$ on computation time is. This is because that the large value of $k$ would leads to a larger nest in traversing for all the $k-\omega$ sets and larger loop when checking the exposure of nodes, hence the computation time for finding all the sets that $k-\omega$ cover each node and determining the sets of sensor with largest exposure is increased considerably.

In conclusion, there may exist a value of $k$ such that its critical coverage approaches the upper bound while the computation time has not been exceptional.

\begin{figure}[h]
	\centering
	\includegraphics[scale=1.]{Hinhanh/kEffect/main.pdf}
	\caption{Effect of $k$ on the critical coverage value of the $(k-\omega)$ barrier}
	\label{fig:}
\end{figure}

\subsubsection{Compare the devised metric with the traditional coverage models}
The figure below illustrates the result of barrier coverage regarding the density computation, which means the coverage of a barrier is calculated as the average of the coverage of every point on the barrier. Practically, the method to evaluate the coverage of a barrier has been proposed in section \ref{baCal}. The assessment takes into consideration 2 attenuated models, which are the Closest-Sensor and the All-Sensor Intensity.

From the above figure, it is obvious that the differential coverage model outperform the tradition ones

\section{Conclusion}
%
This paper addressed the minimal exposure path problem for attenuated sensing model with all mobile sensors. We first have considered and formulated a model of the minimal exposure path problem in all mobile sensor networks; we have proposed the genetic algorithm to solve this issue. In addition, extensive experimental simulations were conducted to validate and to evaluate the proposed model as well as algorithm. The results showed that: the proposed algorithm could be effectively applied to both static and mobile models of wireless sensor networks; the coverage of mobile sensor networks is almost better than the coverage of static sensor networks in case having the same number of sensors.

%\bibliography{mybibfile}
\bibliography{report}
\bibliography{mybibfile}
\end{document}